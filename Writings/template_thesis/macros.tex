% range of a map:
\newcommand{\ran}{\text{ran}}
% span of a set of vectors:
\newcommand{\hull}{\text{span}}
% the automorphism group:
\newcommand{\Aut}[1]{\text{Aut}(#1)}
% gradient, divergence and rotation:
\newcommand{\grad}{\text{grad}}
\renewcommand{\div}{\text{div}}    % standard meaning is dividing sign
\newcommand{\curl}{\text{curl}}
% the identity operator:
\newcommand{\id}{\mathbbm 1}
% the imaginary unit i:
\renewcommand{\i}{i}
% Dirac notation:
\newcommand{\bra}[1]{\langle #1 |}
\newcommand{\ket}[1]{| #1 \rangle}
\newcommand{\bracket}[2]{\langle #1 | #2 \rangle}
%\newcommand{\matrixel}[3]{\bra{#1} #2 \ket{#3}}

\newcommand{\matrixel}[1]{\matrixelexparg#1\relax}
\def\matrixelexparg#1,#2,#3\relax{\langle #1 | #2 | #3 \rangle}

\newcommand{\vac}{\ket{\text{vac}}}   % the vacuum state
% creation and annihilation operators:
\newcommand{\cre}[1]{a^\dagger_{#1}}
\newcommand{\ann}[1]{a_{#1}}
% backslash:
\newcommand{\bs}{\backslash}
% trace:
\newcommand{\tr}{\text{tr}}
% differential:
\renewcommand{\d}{\text{d}}
% signum of a permutation:
\newcommand{\sgn}{\text{sgn}}
% macro for functional derivative: first argument is name of the functional, second is name of the function, third is the function's variable:
\newcommand{\funcderiv}[1]{\funcderivexparg#1\relax}
\def\funcderivexparg#1,#2,#3\relax{\frac{\delta #1[#2]}{\delta #2(#3)}}
% similar but second functional derivative this time:
\newcommand{\secfuncderiv}[1]{\secfuncderivexparg#1\relax}
\def\secfuncderivexparg#1,#2,#3,#4\relax{\frac{\delta^2 #1[#2]}{\delta #2(#3) \delta #2(#4)}}
% Clebsch-Gordan coefficient:
\newcommand{\clebschgordan}[1]{\clebschgordanexparg#1\relax}
\def\clebschgordanexparg#1,#2,#3,#4,#5,#6\relax{
\left(\begin{array}{cc|c}
#1 & #3 & #5 \\
#2 & #4 & #6
\end{array}\right)
}

\newcommand{\comment}[1]{}

\newenvironment{lrcases}
  {\left\lbrace\begin{aligned}}
  {\end{aligned}\right\rbrace}
  
% custom commands for unicode characters:
\DeclareUnicodeCharacter{2212}{\textminus}
\DeclareUnicodeCharacter{03C0}{$\pi$}
\DeclareUnicodeCharacter{00C5}{\AA}
\DeclareUnicodeCharacter{212B}{\AA}
\DeclareUnicodeCharacter{2010}{-}
\DeclareUnicodeCharacter{200A}{}  % hair space
\DeclareUnicodeCharacter{2009}{}  % thin space
\DeclareUnicodeCharacter{03A3}{$\Sigma$}
\DeclareUnicodeCharacter{03B4}{$\delta$}
\DeclareUnicodeCharacter{0394}{$\Delta$}
\DeclareUnicodeCharacter{03B2}{$\beta$}

% Molecular formulas:
\newcommand{\Cosacsac}{{[Co(sacsac)\textsubscript{2}]}}
\newcommand{\Cuammin}{{[Cu(NH\textsubscript{3})\textsubscript{4}]}\textsuperscript{2+}}
\newcommand{\Cuen}{{[Cu(en)\textsubscript{2}]}\textsuperscript{2+}}
\newcommand{\Cugly}{{[Cu(gly)\textsubscript{2}]}}
\newcommand{\Cudtc}{{[Cu(dtc)\textsubscript{2}]}}
\newcommand{\Cumnt}{{[Cu(mnt)\textsubscript{2}]}\textsuperscript{2\textminus}}
\newcommand{\CuCl}{{[CuCl\textsubscript{4}]}\textsuperscript{2\textminus}}
\newcommand{\distCuCl}{$D_{2d}$-\CuCl}
\newcommand{\spCuCl}{$D_{4h}$-\CuCl}
\newcommand{\Cuacetate}{Cu\textsubscript{2}(µ-CH\textsubscript{3}COO)\textsubscript{4}(H\textsubscript{2}O)\textsubscript{2}}
\newcommand{\IrF}{IrF\textsubscript{6}}
\newcommand{\CrF}{{[CrF\textsubscript{6}]}\textsuperscript{3\textminus}}


