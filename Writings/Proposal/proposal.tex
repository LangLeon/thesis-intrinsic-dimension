\documentclass[12pt, a4paper]{article}
%
%
% Eingebundene Pakete
%
%
\usepackage[utf8]{inputenc}
\usepackage[UKenglish]{babel}
\usepackage{blindtext}
\usepackage{amsmath, amsthm, amssymb, dsfont}
\usepackage[lf]{Baskervaldx} % lining figures
\usepackage[bigdelims,vvarbb]{newtxmath} % math italic letters from Nimbus Roman
\usepackage[cal=boondoxo]{mathalfa} % mathcal from STIX, unslanted a bit
\renewcommand*\oldstylenums[1]{\textosf{#1}}
\usepackage[left=30mm,right=30mm, top=20mm, bottom=25mm]{geometry}
%
%
\usepackage{hyperref}
\hypersetup{colorlinks = true, citecolor=blue, linkcolor=blue}
\usepackage{enumerate}
\usepackage{tikz-cd}
\usetikzlibrary{arrows}
\usepackage{verbatim}
\title{Thesis Proposal: Solving Kernel Constraints using the Wigner-Eckart Theorem}
\date{}
\author{Leon Lang}
%
%
% Umgebungen für Theoreme, Definitionen etc.
%
%
% proposition, lemma, corollary, theorem
\theoremstyle{plain}
\newtheorem{pro}{Proposition}[section]
\newtheorem{lem}[pro]{Lemma}
\newtheorem{cor}[pro]{Corollary}
\newtheorem{thm}[pro]{Theorem}
% definition, example
\theoremstyle{definition}
\newtheorem{dfn}[pro]{Definition}
\newtheorem{exa}[pro]{Example}
% remark
\theoremstyle{remark}
\newtheorem{rem}[pro]{Remark}
%
%
% Commands defined
%
%
% commands for well known rings
\newcommand{\N}{\mathds{N}}
\newcommand{\Z}{\mathds{Z}}
\newcommand{\Q}{\mathds{Q}}
\newcommand{\R}{\mathds{R}}
\newcommand{\C}{\mathds{C}}
\newcommand{\K}{\mathds{K}}

\DeclareMathOperator{\diag}{diag}
\DeclareMathOperator{\lin}{Lin}
\DeclareMathOperator{\aut}{Aut}
\DeclareMathOperator{\Dim}{dim}
\DeclareMathOperator{\End}{End}
\DeclareMathOperator{\Hom}{Hom}
\DeclareMathOperator{\Id}{Id}
\DeclareMathOperator{\im}{Im}
\DeclareMathOperator{\inn}{in}
\DeclareMathOperator{\out}{out}


\begin{document}
\maketitle

\tableofcontents

\section{Positioning of Research}

In the last years, steerable CNNs have been developed extensively, mostly in Amsterdam \cite{cohen2016steerable}  \cite{general_theory} \cite{3d_cnns} \cite{gauge_cnns} \cite{general_e2_cnns}. In a nutshell, steerable CNNs work as follows (described here for brevity only for the case of networks over the real numbers):

The network is supposed to process data given as functions $\R^n \to \R^{c}$, where $n$ is the input-dimension. For example, images correspond to the case $n = 2$. Furthermore, a group $G$ is considered that acts on $\R^n$ by rotations, for example $SO(n)$. Then for each layer, the input and output has a certain \emph{type}. That is, the input consists of a function $\R^n \to \R^{c_{\inn}}$ and the output of a function $\R^n \to \R^{c_{\out}}$, and $SO(n)$ acts on $\R^{c_{\inn}}$ and $\R^{c_{\out}}$ by group actions $\rho_{\inn}$ and $\rho_{\out}$. These actions induce an action on the whole signals:
\begin{equation*}
\left[ \rho(g)(f_{\inn})\right](x) \coloneq \rho_{\inn}(g) f_{\inn} (g^{-1}x)
\end{equation*}
The layer is supposed to respect these actions, i.e. to be equivariant with respect to the rotation group. Concretely that means the following:

Let $K: \R^n \to \Hom_{\R}(\R^{c_{\inn}}, \R^{c_{\out}})$ be the kernel that ``maps'' between the layers by convolution. That is for an input $f_{\inn}: \R^{n} \to \R^{c_{\inn}}$, the output $f_{\out}$ is given by
\begin{equation*}
f_{\out}(x) = \left[ K \star f_{\inn}\right](x) = \int_{\R^n} K(y - x) f_{\inn}(y) dy.
\end{equation*}
The equivariance constrained means that for all signals $f_{\inn}$ and for all group elements $g \in G$, rotation should be respected:
\begin{equation*}
K \star \left[ \rho(g)(f_{\inn}) \right] = \rho(g) \left( K \star f_{\inn} \right)
\end{equation*}
(Technically, the constrained should additionally hold for translations, but convolutions are equivariant to translations in general, and so we omit this constraint in the discussion here).

It was shown in \cite{3d_cnns} that a kernel $K$ has this equivariance property if and only if, for all $g \in G$ and $x \in \R^n$, it holds:
\begin{equation*}
K(gx) = \rho_{\out}(g) K(x) \rho_{\inn}(g)^{-1}.
\end{equation*}
In several recent work, most notably \cite{general_e2_cnns}, a basis of the space of these rotation-steerable kernels has been computed, for different symmetry groups $G$, including groups like $D_n, C_n, O(2), SO(2)$ etc. My work will attempt to investigate these solutions in a more unified way, see next section.

\section{Research Question}

As can be seen from the last section, the kernel constraint only constrains the behaviour of the kernel with respect to points that lie in the same orbit of the group, given by the sphere $S^{n-1}$. Thus, for a continuous kernel $K$, one can solve the constraint for each restriction to spheres of different radii separately and then combine the solutions in such a way that continuity is ensured. That is, we look at continuous maps
\begin{equation*}
K: S^{n-1} \to \Hom_{\R}(\R^{c_{\inn}}, \R^{c_{\out}})
\end{equation*}
such that $K(gx) = \rho_{\out}(g) K(x) \rho_{\inn}(g)^{-1}$. We can call the vector space of all such mappings $\Hom_G(S^{n-1}, \Hom_{\K}(\R^{c_{\inn}}, \R^{c_{\out}}))$. This looks suspiciously like \emph{representation operators} considered in physics:

For three $G$-representations $T, U$ and $V$, the set of representation operators is the set of all \emph{linear} equivariant maps $T \to \Hom_{\R}(U, V)$, denoted by the same symbol $\Hom_{G}(T, \Hom_{\K}(\R^{c_{\inn}}, \R^{c_{\out}}))$. The Wigner-Eckart Theorem \cite{wigner-eckart} analyzes the structure of these representation opterators. 

The somewhat vague research question is: \emph{How can the similarity between steerable kernels and representation operators be made precise, and what does this tell us about bases of steerable kernel spaces?}

In the next section, I'm writing down what I already know about this relation. In the last section, I list follow-up questions.



\section{Progress on the above-mentioned question}

\subsection{The Wigner-Eckart Theorem}

If we write ``$V$ is a representation'', without further clarification, then we mean that $V$ is a vector space that comes equipped with a homomorphism $\rho_V: G \to \aut (V)$.

In this section, we state and prove the Wigner-Eckart theorem, which we will use for solving kernel constraints. The treatment essentially follows the basis-independent form in \cite{wigner-eckart}.

The main ingredient for this theorem is Schur's Lemma, see \cite{Jeevanjee}:

\begin{pro}\label{Schur}
Let $V$ and $W$ be irreducible representations over a group $G$ and let $f: V \to W$ be a linear equivariant map. Then either $f$ is zero or an isomorphism.

Furthermore, if the underlying field is the complex numbers $\C$, then the set of endomorphisms, i.e. linear equivariant maps from $V$ to $V$, is isomorphic to $\C$ itself:
\begin{equation*}
\End(V) = \{c \cdot \Id_V \mid c \in \C\} \cong \C.
\end{equation*}
\end{pro}

Furthermore, in order to state the theorem, we need the notion of representations on tensor products and spaces of linear functions between representations:

If $T$, $U$ and $V$ are representations, we can build the tensor product $T \otimes U$ and the space of linear functions $\lin(U, V)$. Both carry a representation:

\begin{equation*}
\rho(g)(t \otimes u) \coloneqq \rho_t(t) \otimes \rho_U(u),
\end{equation*}
and
\begin{equation*}
\left[\rho_{\text{Hom}}(g)\right](f) \coloneq \rho_V(g) \circ f \circ \rho_U(g)^{-1}.
\end{equation*}

\begin{dfn}
Let $T, U$ and $V$ be three representations. Then a representation operator is a linear equivariant map $\phi: T \to \lin(U, V)$.
\end{dfn}

We have the following alternative description of representation operators, proven in \cite{wigner-eckart}:

\begin{pro}\label{correspondence}
There is a $1$–$1$ correspondence between representation operators $\phi: T \to \lin(U,V)$ and linear equivariant maps $\phi': T \otimes U \to V$. This correspondence is given by
\begin{equation*}
\phi'(t \otimes u) = \phi(t)(u).
\end{equation*}
\end{pro}

\begin{thm}\label{theorem}
Let $T, U, V$ be $G$-representations, of which $V$ is assumed to be irreducible. Let $\overline{K}: T \to \lin(U, V)$ be a representation operator. Then $\overline{K}$ is constrained as follows:

Assume that $V$ appears $n$ times as a direct summand in $T \otimes U$, i.e. there is an isomorphism of representations
\begin{equation*}
T \otimes U \cong V^n \oplus W
\end{equation*}
for some other representation $W$ that splits into irreducibles that are all non-isomorphic to $V$ ($n = 0$ is possible and allowed). Let $\rho_i: T \otimes U \to V$ be the corresponding equivariant linear projections, $i = 1, \dots, n$. Then $\overline{K}$ is given by
\begin{equation*}
\overline{K}(t)(u) = \begin{pmatrix}c_1 & \hdots & c_n \end{pmatrix} \cdot \begin{pmatrix} p_1 \\ \vdots \\ p_n \end{pmatrix} (t \otimes u)= \sum_{i = 1}^{n}c_i \left( p_i(t \otimes u) \right)
\end{equation*}
for endomorphisms $c_i: V \to V$ independent of $t$ and $u$. Furthermore, if the underlying field is the complex numbers $\C$, then the $c_i$ are just complex numbers and called reduced matrix elements of the representation operator. 
\end{thm}

\begin{proof}[Sketch of proof]
The idea is to use the correspondence Proposition \ref{correspondence} in order to get an equivalent description of the space of representation operators:

\begin{align*}
\Hom_G(T, \lin(U, V)) & \cong \Hom_G(T \otimes U, V) \\
& \cong \Hom_G(V^n \oplus W, V) \\
& \cong \bigoplus_{i = 1}^{n} \Hom_G(V, V) \oplus \Hom_G(W, V) \\
& = \bigoplus_{i = 1}^{n} \End_G(V).
\end{align*}

In the second isomorphism, the iso $T \otimes U \cong V^n \oplus W$ was used. The third isomorphism just uses that linear equivariant maps can be described on each direct summand individually. The last equality uses that $W$ does not contain $V$ as a direct summand, and so by Schur's Lemma \ref{Schur}, there is no homomorphism $W \to V$. Now the result follows by taking the tuple $(c_1, \dots, c_n) \in \bigoplus_{i = 1}^{n} \End_G(V)$ corresponding to the representation operator $\overline{K}$ under the above isomorphism and explicitly tracing back the isomorphisms from bottom to top to find the form of $\overline{K}$.

The second statement about $c_i$ being complex numbers in the case that the field is $\C$ follows from the second part of Schur's Lemma \ref{Schur}.
\end{proof}

\begin{cor}\label{corollary}
Let $\{ \varphi_i \mid i \in I \}$ be a basis of $\End_G(V)$. Then the compositions $\{ \varphi_i \circ p_j \mid i \in I, j \in \{1, \dots, n\} \}$ form a basis of $\Hom_G(T \otimes U, V) = \Hom_G(T, \lin(U, V))$.
\end{cor}

\begin{proof}
Note that the elements $(0, \dots, 0, \varphi_i, 0, \dots, 0)$ form a basis of $\End_G(V)^n$. From the proof of Theorem \ref{theorem} it follows that they get mapped by an isomorphism to $\varphi_i \circ p_j$, with $j$ being the index containing $\varphi_i$.
\end{proof}



\subsection{Strategy for finding steerable kernel-bases}

In order find bases for steerable kernels, we will in the next subsection prove a Wigner-Eckart theorem for steerable kernel bases that works for all groups that are usually considered. The end-result will be an explicit description of the steerable kernel-bases using exactly three ingredients. These ingredients are:

\begin{enumerate}
\item A basis for the $G$-endomorphisms of all irreps.
\item All Clebsch-Gordan coefficients underlying the decompositions of all tensor products $V_l \otimes V_m$ into irreps.
\item Orthonormal bases of $V_l$ when viewed as a direct summand in $L^2(S)$, where $S = S^n$ is a sphere in most applications.
\end{enumerate}

\subsection{Version of the Wigner-Eckart theorem for general steerable kernels}\label{general_wigner_eckart}

In this section, we provide a version of the Wigner-Eckart theorem specifically for steerable kernels. We formulate it as general as possible, so that we can apply it in many settings. The definitions are thus far a little vague and the proofs therefore only meant as a ``hint'' for how to prove the statements in generality. 

In this section, let $G$ be a compact topological group, for example $SO(2)$, $O(2)$, $C_n$, $D_n$, $SO(3)$, $O(3)$, $SU(1)$, $SU(2)$, $SU(3)$ etc.

Furthermore, assume $G$ acts on a space $S$ like $S^1$ or $S^2$ from the left. Also assume that $S$ carries a space of square-integrable functions $L^2(S)$ with values in a field $\K$, with $\K = \R$ or $\K = \C$. $G$ acts on this space by $(g \cdot f)(s) = f(g^{-1} \cdot s)$. Furthermore, $L^2(S)$ is assumed to be a Hilbert-space by means of the scalar product
\begin{equation*}
\left\langle f, g \right\rangle = \int_{S} \overline{f(s)} g(s) ds.
\end{equation*}
In that formula, the complex conjugation does not do anything if $\K = \R$. 

We also assume that the irreps are given by $(V_m)_{m \in \Z}$ (or indexed by $m \in \N$) and that every irrep appears exactly once in $L^2(S)$ as a direct summand. 

For $m$, let $[m]$ denote the $\K$-dimension of $V_m$. Let $\{Y^m_i \mid i \in \{1, \dots, [m] \}\}$ be an orthonormal standard basis of $V_m \subseteq L^2(S)$, such that the union of all these functions is an orthonormal basis of $L^2(S)$. For example, if $G = SO(2)$, $S = S^1 = \R/{2\pi\Z}$ and $\K = \C$, then these functions are just the characters $Y^m_1 = \chi_{m}$. For $\K = \R$, these function are given by $Y^m_1 = \cos_m$ and $Y^m_2 = \sin_m$ (probably up to some scalar!). For $G = SO(3)$ and $S = S^2$, we obtain the spherical harmonics.

For lack of a better notation, let $\overline{m}$ denote the $\K$-dimension of $\End_G(V_m)$. Denote by $\{ \varphi_r \mid r \in \{1 ,\dots, \overline{m} \}$ a basis of this space.

Now, with somewhat inconvenient notation, we need to define the Clebsch-Gordan coefficients. Thus, assume that we have given two irreps $V_l, V_m$. For a third irrep $V_n$, let $[n,(l,m)]$ denote the number of times $V_n$ appears in a direct sum decomposition of $V_l \otimes V_m$ (This number can be larger than $1$! For example, it turns out that $V_0$ is twice a direct summand of $V_m \otimes V_m$ for $m \geq 1$ and $\K = \R$). Thus, for $n \in \Z$ and $s \in \{1, \dots, [n,(l,m)]\}$ there are copies $V_n^s \subseteq V_l \otimes V_m$ of $V_n$ in the tensor product such that we get an inner direct sum decomposition
\begin{equation*}
V_l \otimes V_m = \bigoplus_{n \in \Z} \bigoplus_{s = 1}^{[n,(l,m)]} V_n^s.
\end{equation*}
Let $Y_k^{s;n}$ be basis elements in $V_n^s$ corresponding to the standard basis elements $Y_k^n$ of $V_n$. Then, we can write the standard basis elements $Y_i^l \otimes Y_j^m$ of $V_l \otimes V_m$ by means of these basis elements as follows:
\begin{equation*}
Y_i^l \otimes Y_j^m = \sum_{n \in \Z} \sum_{s = 1}^{[n, (l,m) ]} \sum_{k = 1}^{[n]} q_{k,(i,j)}^{s; n, (l,m)}Y_k^{s; n}.
\end{equation*}
The indices $q_{k,(i,j)}^{s;n,(l,m)}$ are called the \emph{Clebsch-Gordan coefficients} corresponding to an explicit decomposition of $V_l \otimes V_m$ into irreps.

Note that the Clebsch-Gordan coefficients immediately induce equivariant projections $q^{s;n,(l,m)}: V_l \otimes V_m \to V_n$, given on the basis by
\begin{equation*}
q^{s;n,(l,m)}(Y^l_i \otimes Y^m_j) = \sum_{k = 1}^{[n]} q_{k,(i,j)}^{s;n,(l,m)} Y_k^{n}.
\end{equation*}
Thus, for fixed $s,n,l$ and $m$, $q^{s;n,(l,m)}$ can be viewed as a matrix of shape $[n]\times ([l]\cdot [m])$. If $[n,(l,m)] = 1$, then for convenience we drop the index $s$ and just write $q^{n,(l,m)}$.

Our final ingredient is the following: for $l \in \Z$, let $p_l: L^2(S) \to V_l$ be the canonical projection, given explicitly by
\begin{equation*}
p_l(f) = \sum_{i = 1}^{[l]} \left\langle  f, Y^l_i \right\rangle Y^l_i.
\end{equation*}
To reduce clutter, we denote by $p_l$ also the projection $p_l = p_l \otimes \Id: L^2(S) \otimes V_m \to V_l \otimes V_m$. By means of the correspondence from Proposition \ref{correspondence}, we also view $p_l$ as a homomorphism $p_l: \Hom_G(L^2(S), \Hom_\K(V_m, V_l \otimes V_m))$ when need arises.

\begin{pro}\label{steerable kernels = representation operators}
There is an isomorphism
\begin{equation*}
\begin{tikzcd}
\Hom_G(S, \Hom_{\K}(U, V)) \ar[rr, bend left = 15, "\overline{(\cdot)}"] & & \Hom_G(L^2(S), \Hom_{\K}(U, V)) \ar[ll, bend left = 15, "(\cdot)|_{S}"]
\end{tikzcd}
\end{equation*}
between the space of continuous steerable kernels on the left and the space of continuous representation operators on the right, given by $\overline{K}(f)(u) = \int_{S}f(s)K(s)(u)ds$ and $K'|_{S}(s)(u) = K'(\delta_s)(u)$, where $\delta_s$ is the Dirac Delta function.
\end{pro}

Note that the Wigner-Eckart theorem \ref{theorem} is formulated for three $G$-representations $T, U$ and $V$ and in terms of $V$ being a direct summand of $T \otimes U$. However, as we also saw, we actually use it slightly different. Namely, we have a space $L^2(S) \otimes V_m$ and first project to all the spaces $V_l \otimes V_m$ such that $V_n$ is (possibly several times) a direct summand. Furthermore, we are not interested in the specific form of \emph{one representation operator} but want to parameterize the space of all such operators. This means that we are well-advised to make the basis of representation operators more explisit. This is the content of the following theorem:

\begin{thm}\label{representation operators basis}
Let $V_m$ and $V_n$ be irreps. Let there be the following sets of projections and endomorphisms, as defined above:
\begin{enumerate}
\item $\left\lbrace p_l = p_l \otimes \Id: L^2(S) \otimes V_m \to V_l \otimes V_m \mid l \in \Z \right\rbrace$, is the set of canonical projections.
\item $\left\lbrace q^{s;n,(l,m)}: V_l \otimes V_m \to V_n \mid s \in \{1, \dots, [n,(l,m)]\} \right\rbrace$ is the set of projections from $V_l \otimes V_m$ to $V_n$ that emerge from the Clebsch-Gordan coefficients.
\item $\left\lbrace \varphi_r \mid r = 1, \dots, \overline{n} \right\rbrace$ is a basis of $\End_G(V_n)$.
\end{enumerate}
Then the set of all possible compositions 
\begin{equation*}
\left\lbrace \varphi_r \circ q^{s;n,(l,m)} \circ p_l \right\rbrace
\end{equation*} 
form a basis of $\Hom_G(L^1(S), \lin(V_m, V_n))$ by setting, with abuse of notation, 
\begin{equation*}
(\varphi_r \circ q^{s;n,(l,m)} \circ p_l)(f)(v) = (\varphi_r \circ q^{s;n,(l,m)} \circ p_l)(f \otimes v).
\end{equation*}
Furthermore, the set of compositions $\varphi_r \circ q^{s;n,(l,m)} \circ p_l|_{S}$ forms a basis of steerable kernels $S \to \lin(V_m, V_n)$.
\end{thm}

\begin{proof}
The projections $q^{s;n,(l,m)} \circ p_l$ correspond to the projections in the Wigner-Eckart theorem $\ref{theorem}$ and Corollary \ref{corollary}. Then it follows directly from that Corollary that the compositions $\varphi_r \circ (q^{s;n,(l,m)} \circ p_l) = \varphi_r \circ q^{s;n,(l,m)} \circ p_l$ form a basis of the space of representation operators. Consequently we see from Proposition \ref{steerable kernels = representation operators} that we get a basis of steerable kernels by restricting to $S$.
\end{proof}

The final goal is to get more insight into how the Steerable kernel-basis, given by the compositions $\varphi_r \circ q^{s;n,(l,m)} \circ p_l|_S$, looks like more explicitly. Fortunately, the projections $q$ are already given in matrix-form. We now do the same for $p$:

\begin{cor}
For $s \in S$, let $p_l|_S(s): V_m \to V_l \otimes V_m$. Then, with respect to the standard bases of $V_l \otimes V_m$ and $V_m$, the matrix-elements of this are given by:
\begin{equation*}
\left[ p_l|_S(s) \right]_{(i,j),h} = \begin{cases}
Y_i^{l}(s), \ j = h \\
0, j \neq h
\end{cases}
\end{equation*}
\end{cor}

\begin{proof}
This follows directly from the computation
\begin{align*}
\left[ p_l|_S(s)\right] (Y_h^{m}) & = p_l(\delta_s \otimes Y_h^m) \\
& = p_l(\delta_s) \otimes Y_h^m \\
& = \sum_{i = 1}^{[l]} \left\langle \delta_s, Y_i^l\right\rangle \left( Y_i^l \otimes Y_h^m \right) \\
& = \sum_{i = 1}^{[l]}Y_i^l(s) \left( Y_i^l \otimes Y_h^m\right).
\end{align*}
In the last step, the following computation was used, which follows from the properties of the Dirac-delta:
\begin{equation*}
\left\langle \delta_s, f \right\rangle = \int_{S} \delta_s(s') f(s') ds' = f(s).
\end{equation*}
\end{proof}

We note the following intuitive interpretation of this Corollary: Let $p_l|_S(s) \in \Hom_\K(V_m, V_l \otimes V_m)$ be given by a $([l] \times [m]) \times [m]$-matrix. Let $Y^l(s)$ be the column-vector with entries $Y^l_i(s)$. Then this matrix is given by:
\begin{equation*}
p_l|_S(s) = \begin{pmatrix}
\begin{bmatrix}
Y^l(s) & 0 & \hdots & 0
\end{bmatrix}
&
\begin{bmatrix}
0 & Y^l(s) & 0 & \hdots & 0
\end{bmatrix}
&
\hdots
&
\begin{bmatrix}
0 & \hdots & 0 & Y^l(s)
\end{bmatrix}
\end{pmatrix}
\end{equation*}

Thus far, we have written all projections explicitly in matrix-form. We furthermore assume that also the basis-endomorphisms $\varphi_r: V_n \to V_n$ are given by their corresponding bases, i.e. we have
\begin{equation*}
\varphi_r(Y^n_k) = \sum_{k' = 1}^{[n]} \left( \varphi_r\right)_{k'k} Y^n_{k'}.
\end{equation*}

Finally, we get the explicit matrix-form of the basis-kernels:

\begin{thm}[Basis-Kernels, matrix-form]\label{matrix-form}
The basis-kernels are given by
\begin{equation*}
\varphi_r \circ q^{s;n,(l,m)} \circ p_l|_S(s) = \varphi_r \cdot \begin{pmatrix} 
Y^l(s)^T \cdot q_1^{s;n,(l,m)} \\
\vdots \\
Y^l(s)^T \cdot q_{[n]}^{s;n,(l,m)}
\end{pmatrix},
\end{equation*}
Where each ``dot'' denotes just conventional matrix multiplication.
\end{thm}

Before we prove this, note that the shapes of the involved matrices makes sense:
\begin{enumerate}
\item $Y^l(s)^T$ is of shape $1 \times [l]$ and $q^{s;n,(l,m)}$ of shape $[l] \times [m]$, so they can be multiplied. The result is of size $1 \times [m]$. When stacking all these results, the outcome is of shape $[n] \times [m]$.
\item $\varphi_r$ is of shape $[n] \times [n]$ and thus can be multiplied with the right outcome to something of shape $[n] \times [m]$ again.
\item Shape $[n] \times [m]$ is exactly what we expect for a linear map $V_m \to V_n$, so the outcome is of the required form.
\end{enumerate}

\begin{proof}
Since $\varphi$ does not change in the computation, we only need to engage with the second and third factor. Denote by $\odot$ the elementwise product of two matrices, followed by a sum over all resulting entries. Then we obtain:
\begin{align*}
q^{s;n,(l,m)} \circ p_l|_S(s) & = 
\begin{pmatrix}
q^{s;n,(l,m)}_1 \\
\vdots \\
q^{s;n,(l,m)}_{[n]}
\end{pmatrix} \cdot  \begin{pmatrix}
\begin{bmatrix}
Y^l(s) & 0 & \hdots & 0
\end{bmatrix}
&
\begin{bmatrix}
0 & Y^l(s) & 0 & \hdots & 0
\end{bmatrix}
&
\hdots
&
\begin{bmatrix}
0 & \hdots & 0 & Y^l(s)
\end{bmatrix}
\end{pmatrix}\\
& = 
\begin{pmatrix}
q^{s;n,(l,m)}_{k} \odot \begin{bmatrix}0 & \hdots & 0 & Y^l(s) & 0 & \hdots & 0 \end{bmatrix}
\end{pmatrix}_{k, j = 1}^{[n], [m]} \\
& = \begin{pmatrix}
Y^l(s)^T \cdot q^{s;n,(l,m)}_{k, (-,j)}
\end{pmatrix}_{k, j = 1}^{[n], [m]} \\
& = \begin{pmatrix}
Y^l(s)^T \cdot q^{s;n,(l,m)}_{k}
\end{pmatrix}_{k=1}^{[n]},
\end{align*}
which is exactly what we wanted to show.
\end{proof}

\subsection{Example: Harmonic Networks with Wigner-Eckart}

A basis of steerable kernels $S^1 \to \lin(V_m, V_n)$ for $SO(2)$ over $\C$ is given by the character $\chi_{n-m}$. How can we see this again, using what we learned in the last section? We just determine the three ``ingredients'' that the theory requires:
\begin{enumerate}
\item The endomorphisms $\End_{SO(2)}(V_m)$ have $\Id_{V_m}$ as a basis, due to Schur's Lemma. Thus, we can just ignore these endomorphisms altogether, since postcomposition with the identity doesn't change anything.
\item $V_l \otimes V_m \cong V_{l+m}$, and the basis element $\chi_l \otimes \chi_m$ maps directly to the basis element $\chi_{l+m}$ by this isomorphism. Thus, the only existing Clebsch-Gordan coefficient is just $1$.
\item The orthonormal bases of $V_m$ are just given by the characters $\chi_m$.
\end{enumerate}
Finally, note that $V_{n-m} \otimes V_m \cong V_n$. Thus, $l = n-m$ is the only index such that $V_n$ appears within $V_l \otimes V_m$. Plugging everything into Theorem \ref{matrix-form}, we see that $\chi_{n-m}$ is a basis for steerable kernels.


\section{What's next:}

The following theoretical questions have to be clarified:

\begin{enumerate}
\item How can Proposition \ref{steerable kernels = representation operators} be made precise and proven thoroughly? What properties does the space $S$ need to have for that to work? Also note that the Dirac-Delta is not actually a square-integrable function, so the restriction of representation-operators to steerable kernels is thus-far only of heuristic nature. 
\item The Wigner-Eckart theorem is usually formulated for actual direct sum decompositions. However, $L^2(S)$ is is only the \ref{closure} of the direct sum of irreps. We then need to extend the Wigner-Eckart theorem to this case.
\item We just assumed that $L^2(S)$ is a direct sum of each irrep appearing once. Depending on what spaces $S$ we consider, this assumption may be changed in some way.
\end{enumerate}

Furthermore, we may also want to then explicitly solve the kernel constraint for a new group, like $SU(2)$ or $SU(3)$ and apply the solution to a problem, for example in Physics.


\bibliographystyle{apalike}
\bibliography{literature}

\end{document}